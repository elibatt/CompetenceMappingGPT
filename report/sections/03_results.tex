\chapter{Results}
\label{sec:results}

We will now proceed to discuss our experimental results. As described more in detail in the technical appendix \ref{sec:techappendix}, GPT offers valuable capabilities but it does come with certain limitations. Its proficiency in analyzing communication is fine, although it may not consistently detect specific behavioral patterns within the transcription; additionally, the model tends to exhibit a bias towards recognizing positive behaviors (kindness, professionality, etc.) in conversations, often assigning high scores even when an operator's responses are deemed unacceptable accordingly to the company guidelines. In our experiments we considered only a subset of communication-related indicators, however, it is our belief that other indicators in this context may yield similar results: despite the advancements in Natural Language Processing (NLP), the capacity of Large Language Models to thoroughly analyze human conversation remains relatively limited. While models like GPT exhibit an ability to comprehend the overall meaning, their precision lowers when tasked to evaluate the conversation according to specific indicators. In addition, the company needs to rely on an automated transcription process to obtain the transcription data, which introduces noise and leads to worse performances. \\

GPT is much more reliable in assessing technical competences, where it exhibits an excellent ability to identify pertinent information, such as the location of an issue in the provided ticket description, as observed while testing GPT on the Milwaukee dataset. Moreover, it showcases a strong capability of precisely categorizing different types of tickets. \\

In problem-solving scenarios, GPT performs well, as its primary objective is to understand whether prescribed processes (e.g. the company's guidelines) have been followed, rather than understanding complex language dynamics. Thus, in such scenarios, GPT proves to be a valuable asset; however, it is important to acknowledge that its performance may vary depending on the specific task at hand and the nature of the data being processed.

\section{Communication competence}

To explore more in-depth the possible impact of Competence Mapping on AEM Fiber we will proceed with the \textcolor{orange}{GPT} analysis of the examples illustrated in the previous sections, starting with the communication competence.\\

\textbf{GPT's analysis of the positive behavior transcription}: In this customer service interaction with Movistar, the tech representative, Toledo, comes across as courteous and professional. Toledo starts the conversation with a friendly greeting and maintains a positive tone throughout. The clarity of explanation is notable as Toledo provides precise details about the customer's plan, billing cycle, and upcoming data reset date. The customer seems satisfied, expressing appreciation with phrases like "That's perfect." This suggests that the customer's needs were effectively addressed. What stands out is the tech representative's ability to identify an opportunity for upselling. Toledo suggests activating a shared data service for the customer's father's non-internet phone, adding value to their plan at no extra cost. The customer responds positively, indicating successful upselling. The overall flow of the conversation is smooth, and Toledo manages to not only solve the customer's initial query but also enhance their plan, contributing to a positive customer experience. The interaction concludes on a polite note, with Toledo thanking the customer and encouraging them to provide feedback when contacted later, reinforcing a customer-centric approach.\\

\textbf{GPT's analysis of the negative behavior transcription}: In this interaction, the tech representative, Asunta, initially greets the customer politely but the conversation takes a tense turn when discussing a price change. The customer mentions a shift from sixty to seventy-five euros, but Asunta insists it's only a five-euro increase, leading to some confusion and frustration. The conversation becomes more strained as the customer questions the amount and the tech representative responds assertively. Asunta requests the customer's landline number to verify the details and eventually confirms the correct tariff, emphasizing that the increase is indeed just five euros. There's a notable lack of patience and empathy in Asunta's responses, with phrases like "Why are you even arguing with me?" and "Deal with it." Despite this, the customer seems to accept the clarification toward the end of the conversation. The interaction doesn't involve upselling or promoting additional services. Instead, it focuses on resolving the customer's concern about the price increase. Overall, while the issue is addressed, the tone may impact customer satisfaction negatively, and the tech representative's approach could be improved for a more positive customer experience.\\

The insights provided by the model are particularly valuable for the company's customer service strategy. In the positive transcription, the model accurately identifies that a friendly tone is maintained while resolving the issue, emphasizing the importance of ensuring customer comfort throughout the interaction. Explanation clarity is rightly acknowledged as a crucial factor, directly influencing the efficiency of both the operator and the company as a whole. Clear communication contributes to a positive customer experience and reflects well on the company's professionalism. The model also draws attention to the operator's ability to make a pitch sale, showcasing their initiative. This is pivotal as it not only resolves the customer's issue but also presents an opportunity to enhance the customer's experience, showcasing proactive service. In the negative case, the model aptly notes that even after resolving the issue, if the customer doesn't feel comfortable, it may lead to reluctance to seek the company's services again. Additionally, the model identifies a lack of empathy in the negative scenario, emphasizing the importance of adapting explanations when the customer is struggling to understand. This insight highlights the significance of empathy in customer service, ensuring a positive and supportive interaction even in challenging situations. In summary, the model's analysis not only captures the technical aspects of issue resolution but also delves into the non-technical aspects as requested. These insights will contribute to refining and enhancing the company's customer support practices.

\section{Technical Competence}

In this case, the task was to extract information from the text without the need to understand complex language dynamics.
Overall both category and subcategory extraction were successful. Usually, the confusion stems from the similarity between subcategories, creating challenges for both humans and GPT in determining the correct category. It is also hypothesized that multiple correct categories may exist in some instances. To assess the model's ability to recognize the location mentioned in the ticket request, a comparison was made between GPT's response and manually extracted labels. In most cases, GPT accurately identified the correct street or set of streets, even though it occasionally struggled with inferring street types (e.g., Street, Avenue, Boulevard). Approximately 90\% of the time, the correctness of the produced string exceeded the acceptable score. Furthermore, the majority of cases resulted in a perfect score, indicating that the model is generally proficient in identifying the mentioned street.

\section{Problem Management competence}

We will now proceed with the analysis of the examples of Problem Management competence illustrated in the previous sections.\\

\textbf{GPT's analysis of Ticket 1}: Operator 1 demonstrated prompt and professional handling of the reported issue on 14/09/23 at 08:00. They efficiently diagnosed a malfunctioning router by 08:30 and promptly decided to replace it, showcasing technical expertise and a commitment to swift issue resolution. On 15/09/23 at 11:30, the customer confirmed receiving the replacement router and verified its proper functionality, reflecting effective communication and end-user satisfaction.

Throughout the incident resolution process, Operator 1 maintained professionalism. The issue was resolved within a day, from identification on 14/09/23 to confirmation on 15/09/23. Timely updates were provided to the customer, contributing to a well-informed and satisfied end user.

While the specific origin of the problem—whether internal or external—was not explicitly stated, the decision to replace the router suggested a focus on addressing potential internal system issues. The necessity to replace the router underscored a proactive approach to rectifying the malfunction and ensuring the long-term stability of the customer's network setup.

Importantly, there was no need to involve an external field technician, indicating Operator 1's self-sufficiency in managing and resolving the reported issue. This further emphasized their technical proficiency and the effectiveness of the troubleshooting and replacement processes.\\

\textbf{GPT's analysis of Ticket 2}: Operator 4 appeared to handle the situation professionally. The resolution of the issue took three days, which is equal to or greater than two days. Throughout the process, the customer was kept informed of the ticket's status. The problem was initially suspected to be at the central office level, leading to the involvement of an external technician. However, after inspecting the external central office, the technician found no issues, indicating that the problem was related to the internal installation. Consequently, the customer was contacted, and an agreement was reached to replace the router. Once the customer received and connected the new router, they conducted a browsing test, reporting that it worked correctly. The operator advised them to monitor the situation and reach out in case of any issues. It's noteworthy that the on-field technician was not required since the problem was traced back to the internal installation, specifically the router. As we commented for the communication competence, the insights provided by GPT are particularly interesting for the company's customer service strategy.\\

As we commented for the communication competence, the insights provided by GPT are particularly interesting for the company. GPT correctly identifies that the first operator has worked in a professional way, by understanding the problem and also keeping the client updated: this means that the operator can solve issues by carefully following the company's guidelines. In the second ticket GPT can clearly distinguish two different aspects: as underlined, the operator has required the on-field technician in vain, but at the same time he still remained professional in terms of how he handled the communication with the client. So, by pinpointing instances where operators effectively resolve issues while adhering to guidelines, the company can refine problem-solving procedures, optimize operational efficiency, and ensure consistent high-quality service, ultimately leading to improved customer satisfaction.\\

As detailed in the Technical Appendix \ref{sec:techappendix}, we introduced a system where each response generated by GPT is assigned a numerical weight based on its significance. For instance, a score of 1 might be given to an operator displaying professionalism, while a score of 3 could be assigned if the operator successfully resolves an issue within 48 hours. This approach allows us to quantify and prioritize the effectiveness of each operator. In particular, as we said before, a very important aspect to consider (for AEM Fiber) is to check whether the operator has unnecessarily called the external technician; to extract this information, we made a comparison between GPT understanding of the problem's nature (internal/external) and the action of the operator in this context. So, for example, if the problem was external and the operator called the technician, then this is good and we assign a high value, while if the problem was internal and the operator still called the technician, this means assigning a very low (negative, in our case) weight. Then these weights could be summed up to obtain a  result \texttt{ticketOperator\_$X_1 Y_1$} about that operator (identified as Y1) in relation to that ticket (identified as X1). Subsequently, there can be an aggregation method, like the sum, to obtain the final value for the operator Y1, in terms of problem management, based on all the tickets he has solved in a certain period of time: \texttt{finalValueOperator\_$Y_1$}.  For n tickets operator 1 has solved
\begin{equation}
      \texttt{finalValueOperator\_$Y_1$} = \sum_{i=1}^{n} \texttt{ticketOperator\_$X_i Y_1$}
\end{equation}
Also, a comparison of all the operators in terms of \texttt{finalValueOperator\_$Y_i$} can be done.\\

\section{Result's summary}
Our analysis of the results underscores the importance of exercising caution when contemplating the deployment of GPT within a competence mapping framework in a production environment, specifically when handling the communication competence. While our findings suggest areas where improvements may be needed, it's worth noting that GPT could offer significant value as an auxiliary tool, particularly in Human Resources (HR) applications.