\chapter{Conclusions}


In conclusion, this research paper has presented an examination of a subset of competences within the context of competence mapping in a telecommunication company.
In particular, we chose to test the evaluation skills of the GPT model: in summary, GPT does not excel in evaluating communication abilities and may miss some behavioral patterns. Results demonstrate the model's reliability in extracting information from text, evaluating technical proficiency, and assessing problem-solving capabilities, but its performance can vary depending on the task and data at hand. Therefore using GPT in a company's competence mapping framework should be approached with caution; at the same time, it can be particularly valuable for HR support applications. However, it is important to acknowledge that this work investigates only a fraction of the broader landscape of possible competences, and numerous aspects remain unexplored. Regarding privacy concerns, it can still be considered an issue to be discussed in a legal environment, as some of the data processed by GPT may be sensitive. As technology continues to advance, the potential for more sophisticated and capable AI models becomes increasingly evident: we believe that the new GPT-4.0, with its enhanced capabilities and deeper understanding, could serve as a valuable tool for further investigations in this domain.



