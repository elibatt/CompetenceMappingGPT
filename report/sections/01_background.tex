\chapter{Background}

\section{Introduction}

Monitoring employees effectively is key to reaching company goals and allowing sustainable growth. Effectively monitoring employees is essential for achieving company goals and facilitating sustainable growth. This ranges from assessing whether the workforce is adequate for a certain area to evaluating individual employee performance. A valuable approach to do so, as emerges in literature, is competence mapping. The concept of competence is widely used in various fields, but it lacks a universally accepted definition: in this work, individual competence is defined as the ability to effectively use knowledge, skills, and personal resources to meet performance expectations in work assignments. This definition emphasizes the importance of not only possessing knowledge and skills but also applying them to deliver value to both the organization and the individual. In essence, competence encompasses more than merely possessing the right knowledge and skills; it involves applying them effectively to make meaningful contributions \cite{takey2015}.

\subsection{Benefits}

The benefits of having a competence mapping system in place are two-fold: both the organization as a whole and the individual employees will benefit from it. As previously mentioned, competence mapping equips organizations with a clear understanding of their employee's skills and knowledge, therefore allowing a clearer organization of the workforce. Identifying skill gaps and growth potential within the workforce also allows organizations to tailor training and development programs for each employee: this helps to demonstrate a commitment to their career advancement and helps to retain talented personnel. Furthermore, giving each employee the right task and pairing them with other people possessing complementary skills makes collaboration easier and more effective. Moreover, employees' efforts and contributions become more visible through competence mapping, potentially leading to higher motivation. Finally, identifying skill gaps allows employees to focus on areas in which they need to improve and to stay updated on the latest changes in the industry.\\

We will now delve into the general competence mapping workflow and the main tools within this context. This examination is crucial for comprehending the drawbacks of existing methodologies

\section{Competence mapping workflow}

The following workflow was modeled starting from Wickramasinghe, Vathsala and Zoyza (2009)\cite{srilanka_mapping_workflow}.

\begin{enumerate}
      \item The first step in competence mapping is to identify the organization's specific goals and needs. What competences are needed? What is the final objective? After understanding the potential of each employee, do we want to develop a training program for them?
      \item The second step is to carefully choose the competences to map. Not every competence can be measured and not every employee needs to have the same competences. In each company, there are different roles and each role is related to one or more specific company departments: as much as it could help to have a broad view of all employees' competences, it is fundamental to focus on the most important for each role, determined by asking questions to domain experts or consulting literature. Finally, competences should be adapted to existing standards, for example using the ECompetence Framework when defining digital competences, as specified in \ref{sec:ecompetence-framework}: this standardization plays a pivotal role in providing companies with a clearer understanding of the necessary skills required, and it establishes a shared vocabulary internally, ensuring a unified language for evaluating competences.
      \item Before collecting the necessary data, the next step is to define which Key Performance Indicators (KPIs) to consider to evaluate the chosen competence: for example, the technical competence can be evaluated through the number of completed tasks, the time required to complete each task or the number of errors. Furthermore, it is important to determine the frequency at which each employee should be evaluated: employees should be evaluated more than one time, as people learn with time, maybe even as a consequence of the training programs in place. Continuous updates to the employee profile are needed.
      \item The following step is actually mapping the competences, which will be the focus of this paper. Various techniques are currently available: from internal reviews, self-assessments, and manager evaluation to the use of external tools and websites.
      \item Finally, it's possible to operate on obtained data: hiring new people, reorganizing personnel roles, or changing the structure of the company.
\end{enumerate}

\section{Techniques}

We will now explore more in detail the various techniques available to evaluate employees' performances. The statistical approach involves leveraging company data through platforms like PowerBI \cite{powerBI} for a detailed analysis. This allows organizations to gain insights into employee performance trends, identify strengths, and address areas for improvement starting from real everyday data. Another common method is self-assessment, where employees evaluate their own performance. This subjective perspective can be cross-verified through peer reviews or manager reviews, providing a well-rounded understanding of individual contributions. GFoundry, as discussed in section \ref{sec:gfoundry}, is a tool that facilitates this self-assessment process. Additionally, evaluating the tangible contributions of employees, such as their work products, offers a concrete measure of performance. However, scaling this approach can be challenging due to the diversity of roles and tasks within an organization. To assess hard skills, organizations may utilize questionnaires and tests, either developed internally or through external providers like SkillUp \cite{skillup} or TestGorilla \cite{testgorilla}, as highlighted in section \ref{sec:skillup}. These tests aim to understand technical proficiency and knowledge relevant to specific job roles. Soft skills, equally important in the workplace, can be evaluated through interviews or group activities. Simulated scenarios allow employees to showcase their problem-solving abilities, teamwork, and communication skills.

\subsection{GFoundry}
\label{sec:gfoundry}

GFoundry is a talent management platform that leverages gamification to increase workplace productivity by supporting all talent cycle processes to meet current and future business needs \cite{gfoundry}. There are several evaluation models to assess the performance of employees available in GFoundry:
\begin{itemize}
      \item Top-Down Evaluation
      \item Management by Objectives (MBO)
      \item Self-Assessment
      \item 360º Feedback and Evaluation
\end{itemize}
Different evaluation cycle frequencies can be defined as not every company has the same needs regarding how many employees should be tested. All this information is then made available to the management, through their website, so that they can make decisions according to the performances shown by their team members.

\subsection{SKill-Up}
\label{sec:skillup}

Skillup is an innovative technical, soft, and managerial skills assessment system customizable for all segments of the corporation. It is based on a set of interactive tests linked to behavioral indicators depending on the target and assessment profile \cite{skillup}.
It is possible to configure a specific assessment plan based on the company's competency profile. Differently from GFoundry where the assessment was an evaluation score given by the manager or by self-evaluation, in Skillup, there are various tests, of different, and broken down into tasks in order to assess a specific indicator. As in GFoundry, the complete report is then available to the area manager.

\section{Problems of the current methodologies}


Several challenges are faced with current methodologies. One notable issue is the reliance on time-consuming tests, assessments, and evaluations, requiring a substantial amount of employee time and resulting in a decrease in productivity. Moreover, the length and complexity of these assessments may discourage participation, leading to incomplete or inaccurate competence profiles. Johansson (2019) \cite{johansson2019assess} emphasizes that individuals often misjudge their competences and capabilities. Even when competences are evaluated through concrete tasks, as seen in platforms like Skillup, the data, and situations are simulated, introducing a potential gap between assessment and real-world application. Additionally, some competence mapping methods can be invasive, employing constant observation or screen recording, making employees uncomfortable, and raising privacy concerns, especially in the context of data protection regulations like GDPR \cite{computersweekly_uncomfortableworkers}. Traditional methodologies further exhibit limitations by offering static assessments, providing only a snapshot of an employee's skills at a specific moment. This approach fails to consider ongoing skill development and the dynamic nature of job requirements. Furthermore, a predominant focus on hard skills and technical competences overlooks the equally important soft skills, such as communication abilities. Assessing competences becomes ethically challenging in certain categories, particularly in terms of behavioral competences. Moreover, outsourcing knowledge management to external companies introduces dependencies and potential risks, as organizations may lose control over their data or expose it in the event of a breach. Also, external companies may not fully comprehend the organization's culture and goals, leading to suboptimal competence mapping. In addressing these issues, there is a need for innovative approaches to competence mapping that consider both hard and soft skills, promote continuous assessment, and respect employee privacy and morale. Such approaches should be aligned with organizational culture and goals, ensuring a comprehensive understanding of competences while avoiding the pitfalls associated with current methodologies.

\section{Our proposed framework}
\label{sec:proposedframework}

In response to the challenges associated with current competence mapping methodologies, we focused on an innovative approach that leverages both hard and soft skills, encourages continuous assessment, and minimizes disruption to employees' working hours. Our proposed framework integrates cutting-edge technologies in Natural Language Processing (NLP) to automate the evaluation of workforce competences. Traditionally, competence mapping involves tests, surveys, and scenarios, which can be prone to bias and particularly time-consuming. In contrast, our approach capitalizes on Large Language Models (LLMs), advanced NLP models capable of comprehending diverse textual data and recognizing nuanced situations within the provided information. LLMs offer several advantages:

\begin{itemize}
      \item Versatile Data Analysis: LLMs can analyze a broad spectrum of data, ranging from call transcriptions to work notes, providing a comprehensive understanding of an employee's capabilities.

      \item Contextual Understanding: These models excel at understanding context while analyzing textual data, offering the flexibility to explore and assess soft skills effectively.

      \item Scalability: LLMs can handle large volumes of data efficiently, enabling the assessment of competences across the entire organization.
\end{itemize}

By automating the competence mapping process through LLMs, we address the time-consuming nature of traditional methods. This automation not only accelerates the mapping process but also facilitates quicker decision-making by management. Furthermore, traditional assessments can be invasive and disrupt employees' workflow. Our proposed framework, relying on existing data sources, minimizes invasiveness and seamlessly integrates with employees' daily activities. Automating the competence mapping process also yields cost savings for organizations, reducing the expenses associated with traditional assessments. This, in turn, allows Human Resources and managerial resources to be redirected toward other critical tasks. Unlike traditional methods that often rely on simulated environments, our approach uses real-world data for competency assessments. The continuous and dynamic nature of LLM-based assessments ensures that employees' skill profiles remain up to date. This adaptability aligns with the evolving nature of job requirements and the ongoing skill development of the employee. However, it is crucial to acknowledge privacy concerns associated with processing potentially sensitive data by LLMs. This aspect requires careful consideration in a legal context to establish ethical guidelines and safeguard employee privacy.\\

Our proposed framework thus represents a technology-driven solution to the current challenges in competence mapping.